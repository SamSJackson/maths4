\documentclass{article}
\usepackage[utf8]{inputenc}
\usepackage{indentfirst}
\usepackage{enumitem, amsmath, amssymb}
\usepackage{listofitems}
\usepackage[a4paper, margin=0.8in]{geometry}


\title{Topics in Algebra  --- Feedback Exercise 2}
\author{Samuel Jackson --- 2520998j}
\date{\today}

\begin{document}

\maketitle

%%%
\begin{center}
\section*{Question (1)}
\end{center}

\begin{flushleft}
    Let the dihedral group $D_4$ be denoted as $D_4 = \langle \rho_1, \delta_1 : \rho_1^4 = \delta_1^2 = 1, \, (\delta_1 \rho_1)^2 = 1 \rangle$. This gives us the set $\{1, \rho_1, \rho_2, \rho_3, \delta_1, \delta_2, \mu_1, \mu_2\}$ where $\delta_1, \delta_2$ represent the diagonal symmetries and $\mu_1, \mu_2$ represent the horizontal symmetries. \newline

    The group $H$ generated by the diagonal symmetry $\delta_1$ is $H = \{1, \delta_1\}$. 

    The normaliser $N(H) < G$ and $H \subseteq N(H)$, hence $N(H)$ has order 2, 4 or 8, using Lagrange's. We see that $\rho_1\{1, \delta_1\}\rho_1^{-1} = \{1, \delta_2\} \neq H$. Hence, $\rho_1 \notin N(H)$, so $N(H)$ has order 2 or 4. \newline

    We consider $\delta_2$ and see that $\delta_2\{1, \delta_1\}\delta_2^{-1} = \{1, \delta_1\} = H$. Hence, $\delta_2 \in N(H)$. So $N(H)$ must be order 4. Finally we see that $\rho_2H\rho_2^{-1} = \{1, \delta_1\} = H$, hence we have obtained all our elements in $N(H)$. \newline

    
    We can see that the normaliser of $H$ is $N(H) = \{1, \delta_1, \delta_2, \rho_2\}$. In this case, the normaliser is also normal in $D_4$ (this is not always true) and is isomorphic to the Klein-four group $V_4$. This is clear if you write $N(H)$ as $N(H) = \{1, \rho^2, \delta_1, \rho^2\delta_1\}$.
\end{flushleft}
%%%
\begin{center}
\section*{Question (2)}
\end{center}

\begin{flushleft}
    \begin{enumerate}[label=\roman*)]
        \item False - $\mathbb{Z}$ has no composition series.
        \item True.
        \item True.
    \end{enumerate}

\end{flushleft}
\end{document}
