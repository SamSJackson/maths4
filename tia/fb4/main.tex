\documentclass{article}
\usepackage[utf8]{inputenc}
\usepackage{indentfirst}
\usepackage{enumitem, amsmath, amssymb}
\usepackage{listofitems}
\usepackage[a4paper, margin=0.8in]{geometry}


\title{Topics in Algebra  --- Feedback Exercise 4}
\author{Samuel Jackson --- 2520998j}
\date{\today}

\begin{document}

\newcommand{\order}[1]{|#1|}
\newcommand{\setS}{\mathcal{S}}
\newcommand{\norm}{\trianglelefteq}

\newcommand{\Z}{\mathbb{Z}}

\maketitle

%%%
\begin{center}
\section*{Question (1)}
\end{center}

\begin{flushleft}
	By Theorem 38.11 in Fraleigh, we know that a subgroup of a free abelian group is also a free abelian group. Furthermore, it has a rank less than or equal to 3. \newline 
	
	For example, if $G$ is a free abelian group, then it could be of the form $G \cong \Z \times \Z \times \Z$ - by the Fundamental Theorem of Finitely Generated Abelian Groups.
	Consequently, we have a rank 3 subgroup: $H \cong \Z \times \Z \times 2\Z$, generated by \{$(1,0,0), (0,1,0), (0,0,2)$\}.
	We have a rank 2 subgroup, $H \cong \Z \times \Z$, generated by \{$(1,0), (0,1)$\}.
	We have a rank 1 subgroup, $H \cong \Z$, generated by \{1\}.
	Finally, we have a rank 0 subgroup, the trivial group. 
\end{flushleft}

\begin{center}
	\section*{Question (2)}
\end{center}

\begin{enumerate}[label=\alph*)]
	\item Need to read Theorem 39.12.
	\item Part 2
\end{enumerate}

\end{document}
