\documentclass{article}
\usepackage[utf8]{inputenc}
\usepackage{indentfirst}
\usepackage{enumitem, amsmath, amssymb}
\usepackage{listofitems}
\usepackage[a4paper, margin=0.8in]{geometry}


\title{Topics in Algebra  --- Feedback Exercise 4}
\author{Samuel Jackson --- 2520998j}
\date{\today}

\begin{document}

\newcommand{\order}[1]{|#1|}
\newcommand{\setS}{\mathcal{S}}
\newcommand{\norm}{\trianglelefteq}

\newcommand{\Z}{\mathbb{Z}}

\maketitle

%%%
\begin{center}
\section*{Question (1)}
\end{center}

\begin{flushleft}
	By Theorem 38.11 in Fraleigh, we know that a subgroup of a free abelian group is also a free abelian group. Furthermore, it has a rank less than or equal to 3. \newline 
	
	For example, if $G$ is a free abelian group, then it could be of the form $G \cong \Z \times \Z \times \Z$ - by the Fundamental Theorem of Finitely Generated Abelian Groups.
	Consequently, we have a rank 3 subgroup: $H \cong \Z \times \Z \times 2\Z$, generated by $\{(1,0,0), (0,1,0), (0,0,2)\}$.
	We have a rank 2 subgroup, $H \cong \Z \times \Z$, generated by $\{(1,0), (0,1)\}$.
	We have a rank 1 subgroup, $H \cong \Z$, generated by $\{1\}$.
	Finally, we have a rank 0 subgroup, the trivial group. 
\end{flushleft}

\begin{center}
	\section*{Question (2)}
\end{center}

\begin{enumerate}[label=\alph*)]
	\item Let $G$ be a free group of rank 2, $G = \langle a, b : - \rangle$. Since there is 10 elements in $\Z_{10}$, then there is 10 options for each generator, 100 options total. By theorem 39.12 in Fraleigh, any assignment of generators defines a homomorphism, hence we have $100$ homomorphisms.  
	\item The set of ``onto" homomorphisms is a subset of the ``into" homomorphisms, so we have at most 100 homomorphisms. For this case, we want to avoid the homomorphisms being assigned into subgroups, in order to be surjective. The subgroup of elements $\{0, 2, 4, 6, 8\}$ has 5 elements and consequently contains 25 homomorphisms, as per theorem 39.12. Similarly, the subgroup of elements $\{0, 5\}$ has 2 elements and contains 4 homomorphisms. This would be a total of 29 homomorphisms but the homomorphism of sending $a$ and $b$ to $0$ is being counted twice. Hence, total ``onto" homomorphisms is $100 - (25 + 4 - 1) = 100 - 28 = 72$. There are 72 ``onto" homomorphisms from $G$ to $\Z_{10}$.
\end{enumerate}

\end{document}
