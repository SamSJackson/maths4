\documentclass{article}
\usepackage[utf8]{inputenc}
\usepackage{indentfirst}
\usepackage{enumitem, amsmath, amssymb}
\usepackage[a4paper, margin=0.8in]{geometry}
\usepackage{graphicx, caption}


\title{Differential Geometry  --- Feedback Exercise 1}
\author{Samuel Jackson --- 2520998j}
\date{\today}

\begin{document}

\maketitle

\newcommand{\R}{\mathbb{R}}
\newcommand{\Z}{\mathbb{Z}}
\newcommand{\N}{\mathbb{N}}

%%%
\begin{center}
    \section*{Question (1)}
\end{center}

    \begin{flushleft}
        For the function to be a parameterised curve, we require that that the function is smooth and defined on an open interval. Clearly $\alpha$ is defined on the open interval $(0, 100)$. To determine if $\alpha$ is smooth, we recognise that $\sin$, $\cos$ and polynomials are smooth functions as well as the fact that the composition of smooth functions is smooth. Given $\alpha$ is made up of three smooth functions, we know $\alpha$ is also smooth. \newline  
        
        For $\alpha$ to be regular, we require that $\dot{\alpha}(s) \neq \mathbf{0}$ for all $s \in (0, 100)$. Deriving $\dot{\alpha}$, we find: $\dot{\alpha} = ((2t-1)\cos(t^2 - t), (1-2t)\sin(t^2-t), 2t-1)$. Immediately, we see that for $\dot{\alpha}(\frac{1}{2}) = (0, 0, 0)$, so $\alpha$ is not regular.  
    \end{flushleft}
%%%
\begin{center}
    \section*{Question (2)}
\end{center}

    \begin{flushleft}
        Given the components of $\gamma$ are smooth functions $\sin(s)$ and $\cos(s)$, $\gamma$ is similarly a smooth function on the open interval $\R$. Furthermore, $\gamma$ is regular since $\dot{\gamma}$ is solely comprised of similar $-\sin(s)$ and $\cos(s)$ functions which can not be $0$ simultaneously, for $s \in \R$. Hence, $\gamma$ is a regular parameterised curve (RPC). \newline

        For the RPC $\gamma$, we require that $||\dot{\gamma}|| = 1$ for $\gamma$ to be unit-speed. Hence, we calculate $\dot{\gamma}(s) = (-\sin(s), \dots, -\sin(s), \cos(s), \dots, \cos(s))$, where there is $n$-total $-\sin$ and $\cos$ components respectively. Then, we calculate the magnitude of $\dot{\gamma}$:
        \begin{align*}
            ||\dot{\gamma}(s)|| &= \sqrt{\frac{1}{n}(\sin^2(s) + \dots + \sin^2(s) + \cos^2(s) + \dots + \cos^2(s))} \\
            ||\dot{\gamma}(s)|| &= \sqrt{\frac{1}{n}(n(\sin^2(s) + \cos^2(s))} \\ 
            ||\dot{\gamma}(s)|| &= \sqrt{1} \\ 
            ||\dot{\gamma}(s)|| &= 1
        \end{align*}
        Hence, $\gamma$ is a unit-speed curve. \newline 

        Consequently, we calculate the Frenet-Serret frame for $n=1$. We have $\gamma_1 : \R \rightarrow \R^2$, $s \mapsto (\cos(s), \sin(s))$, so $\dot{\gamma_1} = (-\sin(s), \cos(s)) = \mathbf{T}$. Similarly, since $\gamma_1$ is unit-speed, we can calculate $\mathbf{N}$, which is simple for curves in $\R^2$. $\mathbf{N} = (-\cos(s), -\sin(s))$. Therefore, the Frenet-Serret frame is $\{\mathbf{T}, \mathbf{N}\}$.
    \end{flushleft}
%%%
\begin{center}
    \section*{Question (3)}
\end{center}

    \begin{flushleft}
        To find that $\gamma$ is of unit-speed, we calculate the derivative as $\dot{\gamma} = (\frac{1}{2}\sqrt{1 + s}, \frac{-1}{2}\sqrt{1-s}, \frac{1}{\sqrt{2}})$. Consequently, we calculate the magnitude as follows: 
        \begin{align*}
            ||\dot{\gamma}(s)|| &= \sqrt{\frac{1}{4}(1+s) + \frac{1}{4}(1-s) + \frac{1}{2}} \\ 
            ||\dot{\gamma}(s)|| &= \sqrt{\frac{1}{4} + \frac{s}{4} + \frac{1}{4} - \frac{s}{4} + \frac{1}{2}} \\ 
            ||\dot{\gamma}(s)|| &= \sqrt{\frac{1}{2} + \frac{1}{2}} \\ 
            ||\dot{\gamma}(s)|| &= \sqrt{1} \\ 
            ||\dot{\gamma}(s)|| &= 1 \\ 
        \end{align*}
        Hence, $\gamma$ is unit-speed. \newline

        To calculate the curvature and torsion, we use the respective equations:
        \begin{equation*}
            \kappa = \frac{||\dot{\gamma} \times \ddot{\gamma}||}{||\dot{\gamma}||^3}
        \end{equation*}
        
        \begin{equation*}
                \tau = \frac{\det({\dot{\gamma}} \mid \ddot{\gamma} \mid \dddot{\gamma})}{||\dot{\gamma} \times \ddot{\gamma}||^2}
        \end{equation*}

        As the equations necessitate, we need the second and third derivatives of $\gamma$ which, respectively, are:
        \begin{align*}
            \ddot{\gamma} &= \biggl(\frac{1}{4\sqrt{1+s}}, \frac{1}{4\sqrt{1-s}}, 0\biggl) \\ 
            \dddot{\gamma} &= \biggl(\frac{-1}{8(1+s)^{\frac{-3}{2}}}, \frac{1}{8(1-s)^{\frac{-3}{2}}}, 0\biggl)
        \end{align*}
        Firstly, note that $\gamma$ is a unit-speed curve so $||\dot{\gamma}|| = 1$, hence $\kappa = ||\ddot{\gamma}||$. We solve for the curvature, $\kappa$, first: 

        \begin{align*}
		\kappa &= ||\ddot{\gamma}|| \\ 
		\kappa &= \sqrt{\frac{1}{16(1+s)} + \frac{1}{16(1-s)}} \\ 
		\kappa &= \frac{1}{\sqrt{16}}\sqrt{\frac{1}{1+s} + \frac{1}{1-s}} \\ 
		\kappa &= \frac{1}{4}\sqrt{\frac{2}{1-s^2}} \\ 
		\kappa &= \frac{1}{4}\sqrt{\frac{2}{1-s^2}} \\
		\kappa &= \frac{\sqrt{2}}{4\sqrt{1-s^2}}
        \end{align*}

	For $\tau$, it is a longer calculation. We first calculate $\det({\dot{\gamma}} \mid \ddot{\gamma} \mid \dddot{\gamma})$: 

	\begin{align*}
		\tau &= \det({\dot{\gamma}} \mid \ddot{\gamma} \mid \dddot{\gamma}) \\ 
		\tau &= \dot{\gamma} \cdot (\ddot{\gamma} \times \dddot{\gamma}) \\ 
		\tau &= \dot{\gamma} \cdot \biggl(0, 0, \frac{1}{16(1-s)^\frac{3}{2}}\biggr) \\ 
		\tau &= \frac{\sqrt{2}}{32(1-s^2)^\frac{3}{2}} 
	\end{align*}
	
	Similarly, we calculate $||\dot{\gamma} \times \ddot{\gamma}||$, which is just $||\ddot{\gamma}||$ for a unit-speed curve.
	\begin{align*}
		||\dot{\gamma} \times \ddot{\gamma}|| &= ||\, ||\dot{\gamma}|| \, ||\ddot{\gamma}|| \, \sin(\theta)\mathbf{n}|| \\ 
		||\dot{\gamma} \times \ddot{\gamma}|| &= ||\, 1 \cdot ||\ddot{\gamma}|| \, \cdot 1 \cdot \mathbf{n}|| \\ 
		||\dot{\gamma} \times \ddot{\gamma}|| &= ||\ddot{\gamma}|| \\ 
		||\dot{\gamma} \times \ddot{\gamma}|| &= \frac{\sqrt{2}}{4\sqrt{1-s^2}}
	\end{align*}

	Combining these components, we find the torsion, $\tau$:
	\begin{align*}
		\tau &= \frac{\sqrt{2}}{32(1-s^2)^{\frac{3}{2}}} \cdot \frac{16(1-s^2)}{2} \\
		\tau &= \frac{16\sqrt{2}(1-s^2)}{64(1-s^2)^\frac{3}{2}} \\ 
		\tau &= \frac{\sqrt{2}}{4\sqrt{1-s^2}} 
	\end{align*}
    \end{flushleft}

\end{document}
 