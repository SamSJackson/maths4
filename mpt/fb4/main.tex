\documentclass{article}
\usepackage[utf8]{inputenc}
\usepackage{indentfirst}
\usepackage{enumitem, amsmath, amssymb}
\usepackage[a4paper, margin=0.8in]{geometry}


\title{Measure \& Probability Theory  --- Feedback Exercise 4}
\author{Samuel Jackson --- 2520998j}
\date{\today}

\begin{document}

\maketitle

\newcommand{\N}{\mathbb{N}}
\newcommand{\R}{\mathbb{R}}
\newcommand{\Z}{\mathbb{Z}}
\newcommand{\Q}{\mathbb{Q}}
\newcommand{\I}{\mathbb{I}}

\newcommand{\A}{\mathcal{A}}
\newcommand{\C}{\mathcal{C}}

\newcommand{\borel}{\mathcal{B}(\mathbb{R})}
\newcommand{\powerX}{\mathcal{P}(X)}

\newcommand{\salgebra}{\sigma\text{-algebra}}
\newcommand{\outerset}{\mathcal{M}_{\outermu}}
\newcommand{\outermu}{\mu^\ast}
\newcommand{\outerlamb}{\lambda^\ast}

\newcommand{\invf}{f^{-1}}
\newcommand{\invh}{h^{-1}}
\newcommand{\invg}{g^{-1}}

\newcommand{\nlim}{\lim\limits_{n \rightarrow \infty}}

%%%
\begin{center}
\section*{Question (1)}
\end{center}

\begin{enumerate}[label=(\roman*)]
    \item We define $f(x) = \nlim f_n(x)$. Assume that DCT is true without an integrable function $g \geq 0$ such that $|f_n| \leq g$ for all $n$. Then, by the DCT, $\nlim \int_X f_n d\mu= \int_X \nlim f_n d\mu = \int_X f d\mu$. \newline
    
    We see that $f_n$ is not bounded by an integrable function $g$, for all $n \in \N$, since for any $x \in \R$, there exist an $n \in \N$ such that $f_n(x) = 1$, so $g(x) \geq 1$ for all $x \in \R$. Given that this would not be a finite integral, $g$ is not an integrable function. 
    
    Notice that for all $x \in \R$ and for all $\epsilon > 0$, there is $N = \text{ceil}(x) + 2$ such that for all $n > N$, $|f_n(x) - 0| < \epsilon$. Then, we can see that $f_n$ pointwise converges to $0$, hence $f(x) = 0$ for all $x \in \R$. \newline
    
    For each $n \in \N$, $f_n$ is a characteristic function on a measurable set. Hence, we can calculate integral as the simple measurable function $\int_X f_n d\mu = \int_X  \chi_{[n, n+1]} d\mu = \mu([n, n+1]) = 1$. \newline
    
    Combining this, we see that $\int_X \nlim f_n d\mu = \int_X f d\mu = 0$, while $\nlim \int_X f_n d\mu = \nlim 1 = 1$. Contradicting part (iii) of the DCT.
    
    \item Following from the previous part, we have that $\int_X f_n d\mu = 1$, for all $n \in \N$, while $\int_X f d\mu = 0$. Hence, $\int_X \nlim f_n d\mu = \int_X f d\mu < \nlim \int_X f_n d\mu$.  
\end{enumerate}
    

%%%
\begin{center}
\section*{Question (2)}
\end{center}


\begin{enumerate}[label=(\roman*)]
    \item Since the Riemann integral coincides with Lebesgue integral, provided that the Riemann integral exists, then $\int_A f(x) d\mu$ is the integral of sin$(x)$ on the interval [$0, \pi/2$], which is $-\text{cos}(\pi/2) + \text{cos}(0) = 0 + 1 = 1$. 
    \item Given that the integral over a null-set is a 0 then we can split the integral of this piecewise function. The rationals on the interval $A$ are countable and, hence, are a null-set on the Lebesgue measure. \newline

    We write the integral such as $\int_A g(x) d\mu = \int_{A \setminus \I} g(x) d\mu + \int_{A \setminus \Q} g(x) d\mu$. Due to the piecewise definition of the function, this becomes $\int_{A \setminus \I} \sin(x) d\mu + \int_{A \setminus \Q} \cos(x) d\mu = \int_{A} \cos(x) d\mu$. Then, this integral coincides with the Riemann integral on $[0, \pi/2]$ and $\int_{[0, \pi/2]} \text{cos}(x) = 1$.


    %%% CAREFUL HOW DISCUSSING INTEGRALS OF DOMAINS 
    \item Consider $A = U \cup V$, where $U := \{ x \in A : \cos(x) \in \I \}$, $V := \{ x \in A : \cos(x) \in \Q \}$. Since $U$ and $V$ are trivially disjoint then $\int_A h(x) d\mu = \int_U h(x) d\mu + \int_V h(x) d\mu$. However, $\cos(x)$ is irrational for all non-zero rational values $x$, hence $V = \{ 0, \pi/2 \}$ and $\mu(V) = 0$. Consequently, $\int_A h(x) d\mu = \int_U h(x) d\mu$. \newline 

    By definition of $h$, then $h$ constrained to $U$ is $\sin^2(x)$, hence $\int_A h(x) = \int_U \sin^2(x) d\mu$. Once more, this is a continuous function which coincides with the Riemann integral and --- TO FINISH.
\end{enumerate}

\end{document}
