\documentclass{article}
\usepackage[utf8]{inputenc}
\usepackage{indentfirst}
\usepackage{enumitem, amsmath, amssymb}
\usepackage[a4paper, margin=0.8in]{geometry}


\title{Measure \& Probability Theory  --- Feedback Exercise 2}
\author{Samuel Jackson --- 2520998j}
\date{\today}

\begin{document}

\maketitle

\newcommand{\N}{\mathbb{N}}
\newcommand{\R}{\mathbb{R}}
\newcommand{\Z}{\mathbb{Z}}

\newcommand{\halfopen}[2]{\, [\frac{#1}{100},\frac{#2}{100}) \,}
\newcommand{\borel}{\mathcal{B}(\mathbb{R})}
\newcommand{\powerX}{\mathcal{P}(X)}
\newcommand{\infcup}{\bigcup_{j=0}^\infty}
\newcommand{\infsum}{\sum_{j=0}^\infty}

\newcommand{\outerset}{\mathcal{M}_{\outermu}}
\newcommand{\outermu}{\mu^\ast}
\newcommand{\outerlamb}{\lambda^\ast}


%%%
\begin{center}
\section*{Question (1)}
\end{center}

\begin{flushleft}
    \begin{enumerate}[label=(\roman*)]
        \item We write $A_1$ = $[\frac{2}{10}, \frac{3}{10})$. Similarly, we write $A_2$ as a long chain of unions: $A_2$ = $\halfopen{2}{3} \cup \halfopen{12}{13} \cup \halfopen{22}{23} \cup \halfopen{32}{33} \cup \halfopen{42}{43} \cup \halfopen{52}{53} \cup \halfopen{62}{63} \cup \halfopen{72}{73} \cup \halfopen{82}{83} \cup \halfopen{92}{93}$. The, admittedly long, chain can be simplified into $\bigcup\limits_{i \in I}$ $[\frac{2 + 10i}{100}, \frac{3+10i}{100})$, where $I = \{0, 1, \dots ,9\}$.

        This generates the intuition to find the general form for the set $A_n$. The set $A_n$ can be written as $A_n = \bigcup\limits_{i \in I_n} [\frac{2 + 10i}{10^n}, \frac{3 + 10i}{10^n})$, where $I_n = \{i \in \Z : i < 10^{n-1}\}$.

        Consider the Borel $\sigma$-algebra generated by $\mathcal{B}(\R) = \mathcal{A}(\{[r, \infty) : r \in \R\})$. Let $p \in \N$ and $i \in I_p$, defined previously. Then, by definition of $\borel$, $[\frac{2+10i}{10^p}, \infty)$ and $[\frac{3+10i}{10^p}, \infty)$ are in $\borel$. Furthermore, $\borel$ is a $\sigma$-algebra so $(-\infty, \frac{3+10i}{10^p}) \in \borel$ and $[\frac{2+10i}{10^p}, \infty) \cap (-\infty, \frac{3+10i}{10^p}) = [\frac{2+10i}{10^p},\frac{3+10i}{10^p}) \in \borel$.

        Since $p$ and $i$ were arbitrary, $A_n$ is a union of Borel sets and consequently a Borel set. 
        \item The set $A$ can be written simply as a countable union $A$ = $\bigcup\limits_{n \in \N} A_n$. Then $A$ is a countable union of Borel sets and, therefore, a Borel set.
    \end{enumerate}
\end{flushleft}
%%%
\begin{center}
\section*{Question (2)}
\end{center}

\begin{flushleft}
    \begin{enumerate}[label=(\roman*)]
        \item The outer measure $\outermu : \powerX \rightarrow [0, \infty]$ is defined $\outermu(A) = \text{inf}\{\infsum \mu(E_j) : E_1, E_2, \dots \in \powerX \, \, \text{s.t.} \, \, A \subseteq \infcup E_j\}$, for $A \in \mathcal{P}_f(X)$. \newline

        For a countable set $A \in \mathcal{P}_f(X)$, $A = \bigcup_{a \in A} \{a\}$. Then, using subadditivity, we see that $\outermu(\bigcup_{a \in A} \{a\}) \leq \sum_{a \in A} \outermu(\{a\}) = \sum_{a \in A} \mu(\{a\})$. Hence, $\outermu(A) \leq \sum_{a \in A} \mu(\{a\})$. \newline

        Furthermore, by the definition of $\outermu$, we see that $\outermu(A) \geq \sum_{a \in A} \mu(\{a\})$. Hence, $\outermu(A) = \sum_{a \in A} \mu(\{a\})$, as required.
        

        \item Let $A \in \powerX$, then $A$ is countable since $X$ is countable. Let $S$ be any subset of $X$. Then $\outermu(S \cap A) + \outermu(S \cap A^c) = \sum_{a \in (S \cap A)}\mu(\{a\}) + \sum_{a \in (S \cap A^c)}\mu(\{a\}) = \sum_{a \in S}\mu(\{a\}) = \outermu(S)$. Hence $A \in \outerset$ and since $A$ was arbitrary then $\powerX \subseteq \outerset$. \newline
        
        However, $\outerset \subseteq \powerX$ by definition of $\powerX$, therefore $\outerset = \powerX$.
        \item To show that $\outermu$ is a measure on $\powerX$, we see that $\outermu(\emptyset) = \sum_{a \in \emptyset}\mu(\{a\}) = 0$, as required. We also see that, for any $A \in \powerX$, since $\outermu(A)$ is the sum of $\mu$ over the singletons of $A$ and $\mu$ is a measure, $\outermu(A) \geq 0$. \newline

        Let $A_1, \cdots \in \powerX$ such that the sets are pairwise disjoint then $\bigcup_{i \in \N} A_i \in \powerX$, by definition of the power set. Let $B = \bigcup_{i \in \N} A_i$. Then $\outermu(B) = \sum_{b \in B}\mu(\{b\}) = \sum_{i \in \N}(\sum_{a \in A_i} \mu(\{a\})) = \sum_{i \in \N}\outermu(A_i)$, as required. Hence, $\outermu$ is a measure on $\powerX$.
    \end{enumerate}
\end{flushleft}

%%%
\begin{center}
\section*{Question (3)}
\end{center}

\begin{flushleft}
    \begin{enumerate}[label=\roman*)]
        \item Let $N \subseteq X$ be a null-set and $A \subseteq N$. Then, by Lemma $2.15$, $\outermu(A) \leq \outermu(N) = 0$, hence $\outermu(A) = 0$ and $A$ is also a null-set.
        \item Let $A \subset X$ such that $A$ is a null-set. Then let $S$ be any subset of $X$. Note that $(S \cap A) \subseteq A$ so $\outermu(S \cap A) = 0$. Similarly, since $\outermu(A) = 0$, we can see that $\outermu(S \setminus A) = \outermu(S) - \outermu(A) = \outermu(S)$. \newline

        Hence, $\outermu(S \cap A) + \outermu(S \cap A^c) = \outermu(S)$, satisfying Carath\'{e}odory's condition.
        \item Let $A$ be any countable set, such that $A = \{a_1, a_2, \dots\}$. By the definition of the outer measure, $\outerlamb$, we know that $\outerlamb(A) = \text{inf}\{\,\sum_{k=1}^\infty \lambda(I_k) : A \subseteq \bigcup_{k=1}^\infty I_k \, \}$. Let $\epsilon > 0$, for a given $a_j \in A$, let $I_j = (a_j - \frac{\epsilon}{2^j}, a_j + \frac{\epsilon}{2^j})$. Since $A$ is countable then the union $\bigcup_{j=1}^\infty I_j$ must be countable and $A \subseteq \bigcup_{j=1}^\infty I_j$. \newline

        For any given $k \in \N$, $\lambda(I_k) = \frac{2\epsilon}{2^k}$. Hence $\sum_{k=1}^\infty \lambda(I_k) = \sum_{k=1}^\infty \frac{2\epsilon}{2^k} = 2\epsilon \sum_{k=1}^\infty 2^{-k} = 2\epsilon$. The outer measure $\outerlamb$ is defined as the infimum so $\outerlamb(A) \leq 2\epsilon$ but $\epsilon$ is arbitrary, so $\outerlamb(A) \leq 0$. Therefore, $\outerlamb(A) = 0$ and $A$ is a null-set.
    \end{enumerate}
\end{flushleft}

\end{document}
